\documentclass[11pt]{article}
\usepackage[
a4paper,
margin=1in,
headsep=4pt, % separation between header rule and text
]{geometry}
\usepackage{xcolor}
\usepackage{fancyhdr}
\usepackage{tgschola}
\usepackage{lastpage}
\usepackage[natbibapa]{apacite}


\pagestyle{fancy}
\fancyhf{}
\fancyhead[C]{%
	\footnotesize\sffamily
	\yourname\quad
	web: \textcolor{blue}{\itshape\yourweb}\quad
	\textcolor{blue}{\youremail}}
\fancyfoot[C]{Page \thepage\ of \pageref{LastPage}}

\newcommand{\soptitle}{A computational study of mechanical deformations in abdominal aortic aneurysms}

\newcommand{\yourname}{Debabrata Auddya}
\newcommand{\youremail}{auddya@wisc.edu}
\newcommand{\yourweb}{https://github.com/auddya}

\newcommand{\statement}[1]{\par\medskip
	\textcolor{blue}{\textbf{#1:}}\space
}

\usepackage[
breaklinks,
pdftitle={\yourname - \soptitle},
pdfauthor={\yourname},
unicode
]{hyperref}

\begin{document}
	
	\begin{center}
		\Large\soptitle
	\end{center}
	
	\section*{Motivation}
	An aneurysm is a bulge or "ballooning" in the wall of an artery. Arteries are blood vessels that carry oxygen-rich blood from the heart to other parts of the body. If an aneurysm grows large, it can burst and cause dangerous bleeding or even death. The decision to surgically intercede prior to AA (aortic aneurysm) rupture is made, keeping in view certain procedural risks depending on the maximal diameter or growth rate of the AA. Abdominal aortic aneurysm (AAA), the local dilation of infrarenal aorta [1], is related to the weakening of arterial wall. Therefore, an immediate need to establish a robust and rigorous computational study to better predict rupture risk and improve clinical and surgical measurement integration is inevitable. In this study, we use FEBio software package to model a simplified aortic geometry and allow fluid structure interaction within that structure. While previous literature [2] has focused mainly on the dynamics of fluid and fluid structure interactions within these aneurysms, very trivial attention has been employed to study the solid deformations that undergo. Our goal is to quantify these contortions and present a robust computational model for AA. 
	\section*{Methods}
	Using the FEBio Software package preprocessor, PreView our first task is to create a geometry that resembles an AA, and perform meshing operation. This geometry may either be obtained as a simplified model initially to test preliminary results and later improved, by either using realistic data or importing the mesh from other software. We are making use of a specialized fluid-solid mixture in the FSI module provided by the FEBio software. As a first step to using this module, boundary conditions need to be defined on the fluid and the solid constituents. The experiment is intended to be performed with full Newtonian solver settings in FEBio. Post processing of results is proposed to be performed on PostView. 
	\section*{Objectives and Timeline}
	1.	Modelling simplified geometry of the AA. Building initial mesh (1 week) \\
	2.	Applying boundary conditions. Defining fluid flow parameters. Defining the soft material which makes up the structure of the aneurysm (1.5 weeks)\\
	3.	Implementing FSI routine using FEBio software. Simulating the problem with different categories of flow such as pulsating, streamline, turbulent etc.  Capturing deformation and distortion in the solid during this phase. Post Processing the result. (2 weeks)\\
	4.	Modelling (or importing) realistic aneurysm geometries and performing steps 1-3. (2.5 weeks (tentative))
	\section*{Bibliography}
	1. Lin S, Han X, Bi Y, Ju S, Gu L. Fluid-Structure Interaction in Abdominal Aortic Aneurysm: Effect of Modeling Techniques. Biomed Res Int. 2017;2017:7023078.\\
	2. Chandra S, Raut SS, Jana A, et al. Fluid-structure interaction modeling of abdominal aortic aneurysms: the impact of patient-specific inflow conditions and fluid/solid coupling. J Biomech Eng. 2013;135(8):81001.
	
	
	
	
	
\end{document}