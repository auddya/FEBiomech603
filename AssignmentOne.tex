\documentclass[a4paper,oneside,11pt]{report}
\usepackage[cm]{fullpage}
\usepackage{lmodern,amsmath,amssymb}
\usepackage{a4wide}
\setlength{\marginparwidth}{3cm}
\setlength{\topmargin}{0cm}
\setlength{\voffset}{0cm}
\setlength{\headsep}{0cm}
\title{ME 603 (FE for Biomechanics) - Lab 1}
\author{Debabrata Auddya}
%\usepackage{etoolbox}
%\preto\equation{\setcounter{equation}{0}}
%\makeatletter
%\pretocmd\start@gather{\setcounter{equation}{0}}{}{}
%\pretocmd\start@align{\setcounter{equation}{0}}{}{}
%\pretocmd\start@multline{\setcounter{equation}{0}}{}{}
%\makeatother
\usepackage{listings}
\usepackage{color}
\usepackage{dsfont}
\usepackage{footmisc}
\usepackage{verbatim}
\usepackage{smartdiagram}
\setlength{\marginparwidth}{0cm}
\setlength{\topmargin}{0cm}
\setlength{\voffset}{0cm}
\setlength{\headsep}{0cm}
\definecolor{dkgreen}{rgb}{0,0.6,0}
\definecolor{gray}{rgb}{0.5,0.5,0.5}
\definecolor{mauve}{rgb}{0.58,0,0.82}
\lstset{frame=tb,
	language=Java,
	aboveskip=3mm,
	belowskip=3mm,
	showstringspaces=false,
	columns=flexible,
	basicstyle={\small\ttfamily},
	numbers=none,
	numberstyle=\tiny\color{gray},
	keywordstyle=\color{blue},
	commentstyle=\color{dkgreen},
	stringstyle=\color{mauve},
	breaklines=true,
	breakatwhitespace=true,
	tabsize=3
}
\begin{document}
\maketitle
\section*{1. What step of the FE pipeline is PreView used for?}
In order to set up a finite element problem and perform computations on it, it is important to clearly define the material, its properties and the corresponding mesh. PreView is an FE Preprocessor that allows the user to create or import meshes, specify the boundary conditions and material properties as well as allow the user to choose from a multitude of analysis that is needed for the next steps in the workflow.  
\section*{2. Describe the different modules:}
1. \textbf{Structural Mechanics}:
Mechanics of strcutures is a domain of study within applied mechanics that investigates the response(behaviour) of structures under mechanics loads such as buckling of column, torsion of shaft, deflection of a thin shell, vibration of bridges etc.\\
2. \textbf{Biphasic analysis}:
Pointers: since a
biphasic model can describe the tissue’s experimentally observed creep and stress-relaxation
responses under various loading configurations. \\
3. \textbf{Multiphasic analysis}: An area of analysis which deals with materials with different states or phases, or, materials in the same physical state and variable chemical composition. 
\end{document}
